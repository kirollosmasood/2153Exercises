\documentclass{ximera}
%% handout
%% space
%% newpage
%% numbers
%% nooutcomes

%% You can put user macros here
%% However, you cannot make new environments

\graphicspath{{./}{firstExample/}{secondExample/}}

\usepackage{tikz}
\usepackage{tkz-euclide}
\usetkzobj{all}
\pgfplotsset{compat=1.7} % prevents compile error.

\tikzstyle geometryDiagrams=[ultra thick,color=blue!50!black]
 %% we can turn off input when making a master document

\outcome{Practice with improper integrals.}
\title{The Gaussian Integral}
\author{Kirollos Masood}

\begin{document}
\begin{abstract}
In this activity, we will take a look at a rather tricky improper integral.
\end{abstract}
\maketitle

\begin{sectionOutcomes}
	After completing this section, students should be able to do the following.
	
	\begin{itemize}
		\item State the definition of a function.
		\item Find the domain and range of a function.
		\item Distinguish between functions by considering their domains.
		\item Determine where a function is positive or negative.
		\item Plot basic functions.
		\item Perform basic operations and compositions on
		functions.
		\item Work with piecewise defined functions.
		\item Determine if a function is one-to-one.
		\item Recognize different representations of the same function.
		\item Define and work with inverse functions.
		\item Plot inverses of basic functions.
		\item Find inverse functions (algebraically and graphically).
		\item Find the largest interval containing a given point
		where the function is invertible.
		\item Determine the intervals on which a function has an inverse.
		
	\end{itemize}
\end{sectionOutcomes}

\end{document}


\end{document}
