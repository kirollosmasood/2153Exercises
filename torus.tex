\documentclass{ximera}
%% handout
%% space
%% newpage
%% numbers
%% nooutcomes

%% You can put user macros here
%% However, you cannot make new environments

\graphicspath{{./}{firstExample/}{secondExample/}}

\usepackage{tikz}
\usepackage{tkz-euclide}
\usetkzobj{all}
\pgfplotsset{compat=1.7} % prevents compile error.

\tikzstyle geometryDiagrams=[ultra thick,color=blue!50!black]
 %% we can turn off input when making a master document

%% \outcome{Understand a first example of the Ximera style.}
\title{Solid Torus}

\begin{document}
\begin{abstract}
In this activity, we will use a case study to help us better understand solids of revolution.
\end{abstract}
\maketitle

By now, you may have seen some rather creative shapes constructed by rotating various regions in the $xy$-plane about an axis. Let us take a closer look at yet another.
\begin{definition}
	A \dfn{torus} with radii $R$ and $r$ is the surface obtained by rotating a circle of radius $r$ centered at $(0,R)$ about the $y$-axis.
\end{definition}
We may fill in this circle and rotate the disc around to get a solid of revolution, which we will call a solid torus.
\begin{figure}
	\label{torus}
	\includegraphics{torus.jpg}
	\caption{A graph of a torus.}
\end{figure}

\begin{remark}
	In our definition, we chose to center our circle on the $x$-axis and revolve around the $y$-axis. We could have done it vice versa and gotten the same shape, but oriented differently. We just made a choice so we can set up integrals.
\end{remark} 

\begin{exercise}
	Naturally, we may ask how much volume there is in the new shape we constructed. Let's try to answer this question by setting up an integral with respect to $y$. This means we'll be using the  \wordChoice{\choice{shell method}\choice[correct]{washer method}\choice{nail method}\choice{turtle method}}.
	
	Since $y$ is our independent variable, we'll need to solve for $x$ to get our ``top'' and ``bottom'' functions, or in this case, right and left.
	
	$x_{\mathit{right}}(y) = \answer[given]{R+\sqrt{r^2-y^2}}$
	$x_{\mathit{left}}(y) = \answer[given]{R-\sqrt{r^2-y^2}}$
	
	Now we may set up our integral.
	
	\begin{align*}
		V  &= \int_{-R}^{R} \pi [ (R+\sqrt{r^2-y^2})^{2}-(R-\sqrt{r^2-y^2})^{2}] dy \\
		&= \int_{-R}^{R} \pi [ (R+(r^2-y^2)^{1/2})^{2}-(R-(r^2-y^2)^{1/2})^{2}] dy
	\end{align*}


\end{exercise}

\end{document}
