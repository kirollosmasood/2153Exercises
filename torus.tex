\documentclass{ximera}
%% handout
%% space
%% newpage
%% numbers
%% nooutcomes

%% You can put user macros here
%% However, you cannot make new environments

\graphicspath{{./}{firstExample/}{secondExample/}}

\usepackage{tikz}
\usepackage{tkz-euclide}
\usetkzobj{all}
\pgfplotsset{compat=1.7} % prevents compile error.

\tikzstyle geometryDiagrams=[ultra thick,color=blue!50!black]
 %% we can turn off input when making a master document

\outcome{Practice with solids of revolution.}
\title{Solid Torus}
\author{Kirollos Masood}

\begin{document}
\begin{abstract}
In this activity, we will use a case study to help us better understand solids of revolution.
\end{abstract}
\maketitle

By now, you may have seen some rather creative shapes constructed by rotating various regions in the $xy$-plane about an axis. Let us take a closer look at yet another.
\begin{definition}
	A \dfn{torus} with radii $R$ and $r$ is the surface obtained by rotating a circle of radius $r$ centered at $(0,R)$ about the $y$-axis.
\end{definition}

\begin{image}
	\label{torus}
	\includegraphics{torus2.png}
\end{image}

\begin{comment}
\begin{onlineOnly}
\begin{center}
	\geogebra{rwdexdev}{1200}{600}
	
\end{center}
\end{onlineOnly}
\end{comment}


\begin{remark}
	In our definition, we chose to center our circle on the $x$-axis and revolve around the $y$-axis. We could have done it vice versa and gotten the same shape, but oriented differently. We just made a choice so we can set up integrals.
\end{remark} 

We may fill in this circle and rotate the disc around to get a solid of revolution, which we will call a solid torus.

\begin{exercise}
	Naturally, we may ask how much volume there is in the new shape we constructed. Let's try to answer this question by setting up an integral with respect to $y$. This means we'll be using the  \wordChoice{\choice{shell method}\choice[correct]{washer method}\choice{nail method}\choice{turtle method}}.
	
	Since $y$ is our independent variable, we'll need to solve for $x$ to get our ``bottom'' and ``top'' functions, or in this case, left and right.
	
	$$x_{\mathit{left}}(y) = \answer[given]{R-\sqrt{r^2-y^2}}, \quad  x_{\mathit{right}}(y) = \answer[given]{R+\sqrt{r^2-y^2}}$$
	
	And we will be integrating from $y=\answer[given]{-r}$ to $y=\answer[given]{r}$.
	
	\begin{exercise}
		Now we may set up our integral.
		\begin{align*}
			V  &= \int_{-r}^{r} \pi \left[ \left(R+\sqrt{r^2-y^2} \right)^{2}-\left(R-\sqrt{r^2-y^2} \right)^{2} \right] dy \\
			&= \int_{-r}^{r} 4 \pi R \sqrt{r^2-y^2} dy \\
			&= 4 \pi R \int_{-r}^{r}  \sqrt{r^2-y^2} dy \\
		\end{align*}
		Now let's focus on this last integral above. It exactly computes the area of a \wordChoice{\choice{circle}\choice{sphere}\choice{square}\choice[correct]{semicircle}\choice{icosahedron}}, so this last integral equals $\answer{.5 \pi r^2}$.
		\begin{exercise}
			Now we can put it all together.
			\begin{align*}
				V &= 4 \pi R \int_{-r}^{r}  \sqrt{r^2-y^2} dy \\
				&= 4 \pi R \left[ \frac{\pi r^2}{2} \right] \\
				&= \left[ 2 \pi R \right] \left[ \pi r^2 \right] \\
			\end{align*}
			
			On the other hand, imagine starting out with a plastic rod with a base of $\pi r^2$ and a length of $2\pi R$. Now try to bend it to form a donut (our solid torus). It wouldn't work out, right? It would crack because we started out with something straight and our donut is curved. But if this rod were made out of clay, we could deform it and get back our torus, because the volumes are exactly the same!
			
			We decided to calculate the volume using the washer method. On your own, you may wish to try using the shell method for extra practice. Finally, please select all that apply.
			\begin{selectAll}
				\choice[correct]{I think the torus is cool!}
				\choice[correct]{I want to learn more about geometry.}
				\choice[correct]{I will never think about donuts the same way again.}
				\choice[correct]{I laughed at least once while doing this problem.}
			  \end{selectAll}
			
		\end{exercise}
	\end{exercise}
\end{exercise}


\end{document}
