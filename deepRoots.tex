\documentclass{ximera}
%% handout
%% space
%% newpage
%% numbers
%% nooutcomes

%% You can put user macros here
%% However, you cannot make new environments

\graphicspath{{./}{firstExample/}{secondExample/}}

\usepackage{tikz}
\usepackage{tkz-euclide}
\usetkzobj{all}
\pgfplotsset{compat=1.7} % prevents compile error.

\tikzstyle geometryDiagrams=[ultra thick,color=blue!50!black]
 %% we can turn off input when making a master document

\outcome{Practice with improper integrals.}
\title{Deep Roots}
\author{Kirollos Masood}

\begin{document}
\begin{abstract}
In this activity, we will transcend the limits of man.
\end{abstract}
\maketitle

\begin{exercise}
	Show that
	\begin{align*}
		\sqrt{2}^{\sqrt{2}^{\sqrt{2}^{\rotatebox{45}{\ldots}}  }}=2.
	\end{align*}
	
	Well, let's rephrase the problem using sequences. Define a sequence $\{c_n\}_{n=0}^{\infty}$ as follows.
	\begin{align*}
		c_0 &=\sqrt{2} \\
		c_1 &=\sqrt{2}^{c_0}=\sqrt{2}^{\sqrt{2}} \\
		c_2 &=\sqrt{2}^{c_1}=\sqrt{2}^{\sqrt{2}^{\sqrt{2}}} \\
		&\vdots \\
		c_{n+1}&=\sqrt{2}^{c_n} \\
		&\vdots
	\end{align*}
	So what we actually want to show is that $\lim_{n \to \infty} c_n = \answer[given]{2}$. We can start by just showing the limit exists. We can do this by showing our sequence is (monotonically) increasing and bounded above.
	
	How do we show that $\{c_n\}$ is increasing? We start from the beginning. Is it true that $c_0 \leq c_1$? Well, we know that
	\begin{align*}
		b^x \leq b^y, \quad x \leq y,
	\end{align*}
	assuming that $b \geq 1$. In our situation, our base is $b=\sqrt{2}$, which is bigger than 1, and our exponents are $x=1$ and $y=\sqrt{2}$. So the answer is yes. We've established that $\sqrt{2} \leq \sqrt{2}^{\sqrt{2}}$. Next, we should check if $c_1 \leq c_2$. That means we need to check if 
	\begin{align*}
		\sqrt{2}^{\sqrt{2}} \leq \sqrt{2}^{\sqrt{2}^{\sqrt{2}}}.
	\end{align*}
	But that's true too! Because here, our base is $b=\sqrt{2}$ again. But our exponents now are $x=\sqrt{2}$ and $y=\sqrt{2}^{\sqrt{2}}$. But we just showed that $x\leq y$ a moment ago! We can repeat this process over and over again, which proves our sequence is increasing.\footnote{test}
	
	
	
	
	
\end{exercise}


\end{document}
