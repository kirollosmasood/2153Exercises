\documentclass{ximera}
%% handout
%% space
%% newpage
%% numbers
%% nooutcomes

%% You can put user macros here
%% However, you cannot make new environments

\graphicspath{{./}{firstExample/}{secondExample/}}

\usepackage{tikz}
\usepackage{tkz-euclide}
\usetkzobj{all}
\pgfplotsset{compat=1.7} % prevents compile error.

\tikzstyle geometryDiagrams=[ultra thick,color=blue!50!black]
 %% we can turn off input when making a master document

\outcome{More practice with improper integrals.}
\title{The Gaussian Integral}
\author{Kirollos Masood}

\begin{document}
\begin{abstract}
In this activity, we will take a look at a rather tricky improper integral.
\end{abstract}
\maketitle

Perhaps you have lain awake at night, thinking it is impossible to calculate $\int_{-\infty}^{\infty} e^{-x^2}dx$. After all, we typically evaluate improper integrals by first computing a proper integral and then taking limits. That strategy uses the Second Fundamental Theorem of Calculus, but there is no closed-form expression for $\int e^{-x^2} dx$. Thankfully, we have other methods at our disposal. But we'll need to proceed with finesse since we don't have the heavy machinery of multivariable calculus.

Let's first define a few functions that will help us out.
\begin{align*}
	&F(t)=\int_{0}^{t} e^{-x^2}dx \\
	&G(t)=\int_{0}^{1} \frac{e^{-t^2(1+x^2)}}{1+x^2} dx \\
	$H(t)=F(t)^2+G(t)
\end{align*}
The first function looks relevant, but it's not clear yet how the other ones will help us. Still, let's move on by exploring $H$.
\begin{align*}
H(0)= F(0)^2+H(0)=\left[\int_{0}^{0} e^{-x^2}dx \right]^2+\int_{0}^{1} \frac{1}{1+x^2} dx = \answer[given]{0}^2 + \answer[given]{\pi/4} = \answer[given]{\pi/2}
\end{align*}



















\end{document}
