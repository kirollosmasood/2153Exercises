\documentclass{ximera}
%% handout
%% space
%% newpage
%% numbers
%% nooutcomes

\input{preamble.tex} %% we can turn off input when making a master document

\outcome{More practice with improper integrals.}
\title{The Gaussian Integral}
\author{Kirollos Masood}

\begin{document}
\begin{abstract}
In this activity, we will take a look at a rather tricky improper integral.
\end{abstract}
\maketitle

Perhaps you have lain awake at night, thinking it is impossible to calculate $\int_{-\infty}^{\infty} e^{-x^2}dx$. After all, we typically evaluate improper integrals by first computing a proper integral and then taking limits. That strategy uses the Second Fundamental Theorem of Calculus, but there is no closed-form expression for $\int e^{-x^2} dx$. Thankfully, we have other methods at our disposal. But we'll need to proceed with finesse since we don't have the heavy machinery of multivariable calculus.

\begin{exercise}
	Let's first define a few functions that will help us out.
	\begin{align*}
		&F(t)=\int_{0}^{t} e^{-x^2}dx \\
		&G(t)=\int_{0}^{1} \frac{e^{-t^2(1+x^2)}}{1+x^2} dx \\
		&H(t)=F(t)^2+G(t)
	\end{align*}
	The first function looks relevant, but it's not clear yet how the other ones will help us. Still, let's move on by exploring $H$.
	\begin{align*}
	H(0)= F(0)^2+H(0)=\left[\int_{0}^{0} e^{-x^2}dx \right]^2+\int_{0}^{1} \frac{1}{1+x^2} dx = \left[ \answer[given]{0} \right]^2 + \answer[given]{\pi/4} = \answer[given]{\pi/4}
	\end{align*}
	
	Next, let's find $H'(t)$. We can find the derivative of $F$ using the \wordChoice{\choice[correct]{First Fundamental Theorem of Calculus}\choice{Second Fundamental Theorem of Calculus}\choice{Third Fundamental Theorem of Calculus}\choice{Fundamental Theorem of Algebra}\choice{Fundamental Theorem of Riemannian Geometry}}. Now even though the limits of integration in $H$ don't depend on $t$, the integrand does. So to evaluate $H'(t)$, you may simply pull the derivative (with respect to $t$) inside and treat $x$ like a constant.
	\begin{align*}
		H'(t) =2F(t)F'(t)+H'(t) =2 \left[ \int_{0}^{t} e^{-x^2}dx \right] \left[\answer[given]{e^{-t^2}}\right]+\int_{0}^{1} \frac{ -2t(1+x^2) e^{-t^2(1+x^2)}}{(1+x^2)} dx
	\end{align*}
	
	\begin{exercise}
		Try to simplify the answer above to get
		\begin{align*}
			H'(t) &= 2e^{-t^2} \left[ \int_{0}^{t} e^{-x^2}dx \right] -2e^{-t^2} \left[ \int_{0}^{1} t e^{-t^2 x^2} dx \right] \\
			&=2e^{-t^2} \left[ \int_{0}^{t} e^{-x^2}dx \right] -2e^{-t^2} \left[ \int_{0}^{1} e^{-(tx)^2} (tdx) \right],
		\end{align*}
		where the last change has been made to trigger your desire to integrate by substitution. So we set $u(x)=\answer[given]{tx}$, so that $du=\answer[given]{tdx}$. \\
		We shouldn't forget about our limits of integration! When $x=0$, $u=\answer[given]{0}$. When $x=1$, $u=\answer[given]{t}$.
		
		\begin{exercise}
			Putting it all together, we get
			\begin{align*}
				H'(t)=2e^{-t^2} \left[ \int_{0}^{t} e^{-x^2}dx \right] -2e^{-t^2} \left[ \int_{0}^{t} e^{-(u)^2}du \right]
			\end{align*}
			But $u$ is just a dummy variable, so doesn't that second integral look a lot like $F(t)$? In other words, $H'(t)=0$. That means our function $H$ is constant; it's always $pi/4$. That's a shock!
			
			Now on one hand,
			\begin{align*}
				\lim_{t \to \infty} H(t) = \lim_{t \to \infty} \frac{\pi}{4} = \answer[given]{\pi/4}.
			\end{align*}
			On the other hand,
			\begin{align*}
				\lim_{t \to \infty} H(t) = \lim_{t \to \infty} \left[ F(t)^2+G(t) \right] = \left[\lim_{t \to \infty} F(t) \right]^2 + \lim_{t \to \infty} G(t).
			\end{align*}
			But
			\begin{align*}
				\lim_{t \to \infty} G(t) = \lim_{t \to \infty} \int_{0}^{1} \frac{e^{-t^2(1+x^2)}}{1+x^2} dx = \int_{0}^{1} \frac{ \lim_{t \to \infty} e^{-t^2(1+x^2)}}{1+x^2} dx = \int_{0}^{1} 0 dx = \answer[given]{0}.
			\end{align*}
			This means
			\begin{align*}
				\frac{\pi}{4} = \left[\lim_{t \to \infty} F(t) \right]^2 = \left[\lim_{t \to \infty} \int_{0}^{t} e^{-x^2} dx \right]^2=\left[ \int_{0}^{\infty} e^{-x^2} dx \right]^2.
			\end{align*}
			We may solve to get
			\begin{align*}
				\int_{0}^{\infty} e^{-x^2} dx = \answer[given]{\sqrt{\pi}/2}.
			\end{align*}			
			And finally, since $e^{-x^2}$ is an even function,
			\begin{align*}
				\int_{-\infty}^{\infty} e^{-x^2} dx = \answer[given]{\sqrt{\pi}}.
			\end{align*}
			You may sleep soundly tonight.
			
			
		\end{exercise}
		
	
	\end{exercise}


\end{exercise}


\end{document}
